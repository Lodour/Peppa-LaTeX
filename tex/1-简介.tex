\section{简介}

这是一份 \LaTeX 中文模板手册,包含了常用的预设环境和使用说明。

\subsection{使用方法}

请将模板样式文件\texttt{peppa.sty}和报告源文件\texttt{*.tex}放置在同一目录下,并参照如下样例使用本模板。

\inputminted{latex}{src/template-usage.tex}

\subsection{编译}
使用 \url{https://cn.overleaf.com} 创建项目,并在左侧菜单内设置编译器为 \texttt{XeLaTeX} 即可。

\subsection{字体说明}
默认字体为宋体,其他特殊字体的标识如下。

\begin{itemize}
    \item \textrm{宋体} \latex{\textrm}
    \item \textbf{加粗} \latex{\textbf}
    \item \textsf{黑体} \latex{\textsf}
    \item \textit{楷体} \latex{\textit} \latex{\emph}
    \item \texttt{仿宋} \latex{\texttt}
\end{itemize}

同时,可以使用 \latex{\tiny}, \latex{\scriptsize}, \latex{\footnotesize}, \latex{\small}, \latex{\normalsize}, \latex{\large}, \latex{\Large}, \latex{\LARGE}, \latex{\huge}, \latex{\Huge} 对字体大小进行调整。

\subsection{中文解决方案说明}
根据对 \LaTeX 中文环境的需求程度不同,大致有以下两种解决方案。

\subsubsection{支持中文字符}
这是最常见的需求(如本手册),解决方案如下:

\begin{itemize}
    \item 在导言处添加 \texttt{xeCJK} 宏包
    \item 编译器选择 \texttt{XeLaTex}
    \item 含有中文字符的 \texttt{*.tex} 文件字符编码为 \texttt{UTF-8}
\end{itemize}

\subsubsection{支持更高级的中文样式}

对于更高级的中文样式,如根据 \texttt{Microsoft Word} 文档所规定的字号,或其他更严格的中文样式定制等,推荐使用 \texttt{ctex} 宏包提供的中文文档框架,具体使用方法详见其官网 \url{http://ctex.org}。
